%!TEX root = paper.tex
%%%%%%%%%%%%%%%%%%%%%%%%%%%%%%%%%%%%%%%%%%%%%%%%%%%%%%%%%%%%%%%%%%%%%%%%%%%%%%%%
\begin{abstract}

%stagnate in terms of commercial success. 
In recent years, Cloud Gaming has become a popular research topic, 
and the list of its supposed benefits is long. However, 
Cloud Gaming platforms are still waiting for the commercial breakthrough. 
This might be caused by the pricing 
models and product offerings by existing platforms. This paper aims at investigating the 
costs and benefits of these platforms through a twofold approach.

We first take on the perspective of the customers, and investigate several 
platforms and their pricing models in comparison to the costs of 
platforms that sell games for downloading.
Then, we explore engagement metrics in order to assess the enjoyment of playing the offered games.
Lastly, coming from the perspective of the service providers, we aim to 
give reasons for the problematic nature of operating a large-scale 
Cloud Gaming service while maintaining high \acrshort{QoE} values.

Our analysis provides initial, yet still comprehensive, reasons and 
models for the prospects of Cloud Gaming in a highly competitive market.

\end{abstract}
