%!TEX root = paper.tex
%%%%%%%%%%%%%%%%%%%%%%%%%%%%%%%%%%%%%%%%%%%%%%%%%%%%%%%%%%%%%%%%%%%%%%%%%%%%%%%%
\begin{abstract}

In recent years, Cloud Gaming has become a popular research topic, 
and the list of its supposed benefits is long. However, 
Cloud Gaming platforms stagnate in terms of commercial success. 
Explanations for this circumstance might lie in the services' pricing 
models and their offerings. This paper aims to investigate the 
costs and benefits of these platforms through a twofold approach.

We first take on the users' perspective, and investigate several 
platforms and their pricing models in comparison to the costs of 
platforms that sell games for downloading.
Then, we explore engagement metrics to meter the players' enjoyment.
Lastly, looking from the service providers' perspective, we aim to 
give reasons for the problematic nature of operating a large-scale 
Cloud Gaming service while maintaining high \acrshort{QoE} values.

Our analysis provides initial, yet still comprehensive, reasons and 
models for the prospects of Cloud Gaming in a highly competitive market.

\end{abstract}
