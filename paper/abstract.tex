%!TEX root = paper.tex
%%%%%%%%%%%%%%%%%%%%%%%%%%%%%%%%%%%%%%%%%%%%%%%%%%%%%%%%%%%%%%%%%%%%%%%%%%%%%%%%
\begin{abstract}

The disparity of the popularity of Cloud Gaming in research compared to 
the stagnant commercial success of commercial Cloud Gaming platforms 
has been a common motif in recent years. Explanations as to why this 
circumstance exists are scarce, but may be rooted in the services' 
pricing models and their offerings. This paper aims to investigate the 
costs and benefits of these platforms in a twofold approach.

From the users' perspective an investigation of several platforms and 
their pricing models is conducted in comparison to the costs of 
conventional gaming platforms. To meter the players' enjoyment 
engagement metrics are employed. And looking from the service 
providers' perspective we aim to give reasons for the problematic 
nature of operating a large-scale Cloud Gaming service while 
maintaining high \acrshort{QoE} values.

All in all, this should serve to give some initial, yet still 
comprehensive, reasons and models for the prospects of Cloud Gaming in 
a highly competitive market.

\end{abstract}