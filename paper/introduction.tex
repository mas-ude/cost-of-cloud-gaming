%!TEX root = paper.tex
%%%%%%%%%%%%%%%%%%%%%%%%%%%%%%%%%%%%%%%%%%%%%%%%%%%%%%%%%%%%%%%%%%%%%%%%
%%%%%%%

\section{Introduction}

Among computer applications, video games occupy a special spot.
In contrast to many other software applications, they are almost
exclusively used in a leisure context.
In contrast to consuming media like video (movies, series) and
audio (music, audio books), they are also interactive.
Even for small, independent developers it is possible to develop successful video games. This is mainly due to the availability of third-party engines and game assets, but also not least because of the shift to digital distribution platforms that is already well underway. The result is a large and fast-growing catalog of video games.
But what is less clear are the motivational and engagement factors surrounding the consumers of such video games. Why, for example, is one game picked up over another? More often than not this has less to do with technical details but relies on, e.g., social factors.
So, in order to appropriately determine \gls{QoE} of video games such factors, including context but interactivity as well, need to be regarded in addition to a proper classification of video games by their gameplay aspects.
The recently published \acrshort{ITU-T} Recommendation
G.1032~\cite{itutg1032} follows this notion and discusses influence
factors on gaming \gls{QoE}.

This paper explores such subjective motivational aspects of video gamers
by means of an online survey, and contrasts them with utility
metrics derived from from objective data from several gaming platforms.
The survey results underscore that users value experience-related
characteristics for picking their purchase, and prefer individual choice
over curated catalogs. While an attractive price is an important
reason for buying games in general, personal interest (or lack thereof) and social aspects might be equally important, if not even more so.
% Letzteres ist ziemlich ungeschickt ausgedrückt. Hoffentlich
% finden wir bessere und verständlichere Take-Aways :-)
The objective data provides insight into the offerings of \steam, a digital distribution platform for PC games, and both \psnow and \gfnow, which are cloud gaming services offered by Sony and NVIDIA, respectively.
Of specific interest as utility and engagement metrics are, e.g. the number
of games on offer, their lengths, and review scores.
\todo[inline]{Share a conclusion or two, especially how subj. and obj. data combines.}
\todo[inline]{Quick structure recap of doc.}
%\steam data exhibit how the approximate playthrough length of games
%varies with the price range.
