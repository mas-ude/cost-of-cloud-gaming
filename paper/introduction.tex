%!TEX root = paper.tex
%%%%%%%%%%%%%%%%%%%%%%%%%%%%%%%%%%%%%%%%%%%%%%%%%%%%%%%%%%%%%%%%%%%%%%%%%%%%%%%

\todo[inline,color=red!10]{Other title could be: The Prospects of Cloud Gaming: Do ... }

\section{Introduction}

Cloud gaming has become quite a popular term in recent years in research. Much of this research is aimed at comparing the experienced quality of cloud gaming to that of conventional gaming approaches with the results generally in favor of conventional gaming, but only by a tiny margin. Some publications even praise the method and attest high \gls{QoE} values.

So, if there are only negligible quality drawbacks --- according to the literature --- what about the commercial success of cloud gaming? In theory, there could be large economic benefits for users that should foster a quick adaptation of such services. But looking back to reality, quite a few commercial services have already come and gone, with a rather high rate of fluctuation. For example, one of the most prominent services in the past, OnLive, shut down and was forced to sell off its remaining assets. All of these services, current as well as past, come with some subtle differences in their service, including the technical aspects, the selection of games, as well as the pricing model.

However, none of these seems to have garnered much public interest. Some of this can be attributed to a circumstance that is often overlooked, when evaluating cloud gaming services: the competition 

%%%%%%%%%%%%
\subsection{Fragestellungen}

\begin{itemize}
	\item Kostenmodell für Cloud Gaming Provider?
	\item Attraktivität für “Core Gamer”?
	\item Wieviel ist eine NutzerIn bereit für einen Streaming Service mit einem bestimmten Spieleangebot und einer bestimmten Streaming- Qualität (Video-Qualität, Latenz, Grafikeinstellungen des Spiels) zu zahlen?
	\item Can they be competitive against other gaming platforms, both from provider as well as customer perspective
			Most challenging: can it beat the Steam price model and quality/number of games? (plus bundle providers and sales)
\end{itemize}

Netflix-Analogie? 