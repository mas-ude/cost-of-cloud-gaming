%!TEX root = paper.tex
%%%%%%%%%%%%%%%%%%%%%%%%%%%%%%%%%%%%%%%%%%%%%%%%%%%%%%%%%%%%%%%%%%%%%%%%
%%%%%%%

\section{Introduction}

Among computer applications, video games occupy a special spot.
In contrast to many other software applications, they are almost
exclusively used in a leisure context.
In contrast to consuming media like video (movies, series) and
audio (music, audio books), they are also interactive.
Even for small, independent developers it is possible to develop successful video games. This is mainly due to the availability of third-party engines and game assets, but also not least because of the shift to digital distribution platforms that is already well underway. The result is a large and fast-growing catalog of video games.
But what is less clear are the motivational and engagement factors surrounding the consumers of such video games. Why, for example, is one game picked up over another? More often than not this has less to do with technical details but relies on, e.g., social factors.
So, in order to appropriately determine \gls{QoE} of video games such factors, including context but interactivity as well, need to be regarded in addition to a proper classification of video games by their gameplay aspects.
The recently published \acrshort{ITU-T} Recommendation
G.1032~\cite{itutg1032} follows this notion and discusses influence
factors on gaming \gls{QoE}.

This paper explores subjective motivational aspects of video game
consumers through an online survey, and contrasts them with utility
metrics applied to objective data from different gaming platforms
that are based on PC, console, and cloud gaming.
The survey results underscore that users value experience-related
characteristics for picking their buys, and prefer individual choice
over curated catalogs. While an attractive price is an important
reason for buying games in general, newly-published games are rather
bought (or disregarded) due to personal interest (or lack thereof),
not because of pricing.
% Letzteres ist ziemlich ungeschickt ausgedrückt. Hoffentlich
% finden wir bessere und verständlichere Take-Aways :-)
The objective data provide insight into the offers of \steam
(a digital distribution platform for PC games), \psnow and \gfnow
(console- and streaming hardware based cloud gaming offers by Sony
and NVIDIA, respectively): the number
of games offered, game ages, playthrough lengths, and review scores.
\steam data exhibit how the approximate playthrough length of games
varies with the price range.
