%!TEX root = paper.tex
%%%%%%%%%%%%%%%%%%%%%%%%%%%%%%%%%%%%%%%%%%%%%%%%%%%%%%%%%%%%%%%%%%%%%%%%
%%%%%%%

\section{Introduction}

Among computer applications, video games occupy a special spot.
In contrast to many other software applications, they are almost
exclusively used in a leisure context.
In contrast to consuming media like video (movies, series) and
audio (music, audio books), they are also interactive.
Also, in contrast to the production of movies and the like, video game
production is accessible not only to big studios, but even to lone
enthusiasts, resulting in a very large available catalog of video games.
To complement, there are nicely-working integrated digital distribution
and sales platforms.
These factors suggest that motivation for and engagement in video games
have roots more peculiar and less exclusively technical than
non-interactive or non-leisure applications have, especially since
the large and conveniently accessible offer make switching to other
games easy.
Consequently, the influences on \gls{QoE} of video games should be
interpreted more broadly and also consider context and interactivity
aspects, besides classifications of the actual playthrough course of
games.

This paper explores subjective motivational aspects of video game
consumers through an online survey, and contrasts them with utility
metrics applied to objective data from different gaming platforms
that are based on PC, console, and cloud gaming.
The survey results underscore that users value experience-related
characteristics for picking their buys, and prefer individual choice
over curated catalogs. While an attractive price is an important
reason for buying games in general, newly-published games are rather
bought (or disregarded) due to personal interest (or lack thereof),
not because of pricing.
% Letzteres ist ziemlich ungeschickt ausgedrückt. Hoffentlich
% finden wir bessere und verständlichere Take-Aways :-)
The objective data provide insight into the offers of \steam
(a digital distribution platform for PC games), \psnow and \gfnow
(console- and streaming hardware based cloud gaming offers by Sony
and NVIDIA, respectively): the number
of games offered, game ages, playthrough lengths, and review scores.
\steam data exhibit how the approximate playthrough length of games
varies with the price range.
