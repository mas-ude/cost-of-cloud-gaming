%!TEX root = paper.tex
%%%%%%%%%%%%%%%%%%%%%%%%%%%%%%%%%%%%%%%%%%%%%%%%%%%%%%%%%%%%%%%%%%%%%%%%
%%%%%%%

\todo[inline,color=red!10]{Other title could be: The Prospects of Cloud 
Gaming: Do ... }

\section{Introduction}

Cloud gaming has become quite a popular term in recent years in 
research. Much of this research is aimed at comparing the experienced 
quality of cloud gaming to that of conventional gaming approaches with 
the results generally in favor of conventional gaming, but only by a 
tiny margin. Some publications even praise the method and attest high 
\gls{QoE} values.

So, if there are only negligible quality drawbacks --- according to the 
literature --- what about the commercial success of cloud gaming? In 
theory, there could be large economic benefits for users that should 
foster a quick adaptation of such services. But looking back to 
reality, quite a few commercial services have already come and gone, 
with a rather high rate of fluctuation. For example, one of the most 
prominent services in the past, OnLive, shut down and was forced to 
sell off its remaining assets. All of these services, current as well 
as past, come with some subtle differences in their service, including 
the technical aspects, the selection of games, as well as the pricing 
model.

However, none of these seems to have garnered much public interest. 
Some of this can be attributed to a circumstance that is often 
overlooked, when evaluating cloud gaming services: the strong 
competition with other non-cloud gaming platforms, such as the 
individual console platforms or the large PC market --- with the 
Steam\footnote{\url{http://store.steampowered.com/}} platform as on of 
its strongest contenders. Through the move to digital distribution the 
PC platform has become quite popular and PC games pricing has become 
much more dynamic and affordable in the process.

On the surface, current Cloud Gaming services attempt to a adopt a 
fixed fee subscription model over the traditional à la carte model. 
This move proved to be hugely successful for other types of media, be 
it for example \textit{Netflix} for movies and shows or 
\textit{Spotify} for music. However, these two types of services seem 
to offer much more content at a comparable or even lower price point 
than Cloud Gaming services do. Additionally, the technical requirements 
to maintain a quality level on par with that of locally run games are 
also rather demanding.

The two main questions that this work aims to tackle are thus: 
\textit{``Can Cloud Gaming be attractive for users in today's highly 
competitive market?''} and \textit{``Can you operate a Cloud Gaming 
service with acceptable margins while maintaining acceptable quality 
levels?''}. Both questions are strongly intertwined as in order to make 
such services attractive one would have to offer sufficient quality and 
quantity of games with a competitive pricing while not operating at a 
loss. In practice one could actually simplify both questions quite 
easily to one: \textit{``Can you compete with the PC gaming and Steam 
ecosystem (in terms of quality, prices, and variety)?''}

In order to answer these questions, this paper looks at the perspective 
of users and service provider in separate and provides arguments backed 
by data and simple models. To investigate the customer's perspective we 
employ user domain-specific user engagement metrics --- amongst others 
review scores as well as the length and playtimes of games --- to 
compare various services --- Cloud Gaming as well as conventional --- 
to each other. Additionally, using this data some simple budget models 
are set up to compare what (in terms of the number and quality of 
games) for a certain amount of money. We find that in the investigated 
cases the Cloud Gaming services' offer is very limited yet still 
charges relatively high prices, thus limiting the attractiveness for 
users in comparison to alternative services.

Due to the limited amount of freely available data on operating a Cloud 
Gaming service, the perspective of the Cloud Gaming operator is 
investigated by setting up efficiency models centered around the 
overbooking rate. These initial figures highlight the problematic 
nature of Cloud Gaming in terms of scaling and cost efficiency. When 
compared to other cloud services, that achieve high values of cost 
efficiency and capacity utilization, we believe that Cloud Gaming 
platforms will be much more peak-oriented and can thus achieve much 
lower values of server utilization. The end-to-end lag requirements of 
games demand a server placement in the vicinity of the user and thus 
eliminate most multiplexing gains that a centralized data center can 
garner over the course of a day. Additionally, games require dedicated 
hardware support, which is of no much use to most other cloud use 
cases, diminishing the potential of cross-service reuse.

These initial insights do not shed a good light on the commercial 
future of Cloud Gaming services in general. Apart from achieving major 
cost reductions, while maintaining or better even improving streaming 
quality, the future of Cloud Gaming might be bleak. But there still 
might be some niches to place a Cloud Gaming service where the 
competition is less strong. We plan to take a deeper look at all these 
aspects and provide more detailed models in the future.

~\\
This paper is structured as follows. Sec.~\ref{sec:relatedwork} 
provides a brief overview of the related work. Afterwards, 
Sec.~\ref{sec:background} explains all the necessary terms and 
technical details to understand the topic. The main part of this work 
encompass Sections \ref{sec:engagement} and 
\ref{sec:suppliermodelling}, which conduct the dual-perspective 
investigation of the Cloud Gaming providers' and their competitions' 
business models and engagement from the angle of the user and the 
platform operator respectively. The paper concludes in 
Sec.~\ref{sec:conclusion} with some remarks and an outlook.

%%%%%%%%%%%%
% \subsection{Fragestellungen}

% \begin{itemize}
% 	\item Kostenmodell für Cloud Gaming Provider?
% 	\item Attraktivität für “Core Gamer”?
% 	\item Wieviel ist eine NutzerIn bereit für einen Streaming 
Service mit einem bestimmten Spieleangebot und einer bestimmten 
Streaming- Qualität (Video-Qualität, Latenz, Grafikeinstellungen des 
Spiels) zu zahlen?
% 	\item Can they be competitive against other gaming platforms, 
both from provider as well as customer perspective
% 			Most challenging: can it beat the Steam price 
model and quality/number of games? (plus bundle providers and sales)
% \end{itemize}
% Netflix-Analogie? 