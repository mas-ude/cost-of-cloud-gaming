%!TEX root = paper.tex
%%%%%%%%%%%%%%%%%%%%%%%%%%%%%%%%%%%%%%%%%%%%%%%%%%%%%%%%%%%%%%%%%%%%%%%%
%%%%%%%

\section{Introduction}

Cloud gaming has become quite a popular research topic in recent years.
Much of this research is aimed at comparing the experienced quality of
cloud gaming to that of conventional gaming approaches, often with
reasonable results for the streaming approach (cf., for example,
\cite{5976180}). So, if there are only negligible quality drawbacks,
what about the commercial success of cloud gaming? Intuitively, one
would assume that this could yield substantial benefits in terms of cost
and flexibility as a result of scaling effects through the used cloud
gaming hardware in comparison to equipment-heavy classical home gaming
approaches. However, the cloud gaming market seems to stagnate with a
high rate of fluctuation on the market, i.e., a constant stream of
market entrances and exits. For example, one of the most prominent
services in the past, \textsc{OnLive}, ceased to exist in 2015.

%shut down and was forced to sell off its remaining assets.

%\todo[inline]{PZ: ``sell off its remaining assets'' -- means?}
%\todo[inline]{FM: intellectual property, mainly the remaining patent portfolio (to Sony)}

%In theory, there could be large economic benefits for users that should foster a quick adaptation of such services. quite a few commercial services have already come and gone, with a rather high rate of fluctuation.

Many cloud gaming approaches that vary by means of technical, service
and pricing model differences, have been tested so far, but the public
interest remains low. This might be attributed to the broad range of
available, established substitutes, e.g., non-cloud gaming platforms
such as video game consoles or PCs ---
with\steam\footnote{\url{http://store.steampowered.com/}} as one of the
largest contenders. The move to digital distribution made gaming on PCs
quite popular, and PC games pricing became much more dynamic and
affordable in the process.

On the surface, current cloud gaming services attempt to adopt a
\textit{fixed fee subscription} model over the traditional \textit{à la
carte} model. Fixed fee subscription proved to be hugely successful for
other types of media, e.g., \textsc{Netflix} for movies and shows or
\textsc{Spotify} for music. However, these two types of services offer a
much larger catalogue of content at a comparable or even lower price
than cloud gaming services. Additionally, streaming asynchronous
non-interactive media is technically less demanding (and thus cheaper to
operate) than maintaining a quality level on par with that of locally
running games.

\todo[inline]{Research question: What utility do existing \gls{cg}
platforms offer to consumers, as compared to gaming platforms for
consoles and PCs?}
\todo[inline]{Perspective: user-oriented utility metrics}
\todo[inline]{Based on combining and investigating large public datasets of user choice and game properties.}

The contributions of this paper are as follows.
We characterize the offers of platforms based on properties such
as the number of offered games, play-through length, review scores,
and price.
We investigate cloud gaming platforms, and discuss the various
business models offered by these platforms over the last X years.
Lastly, we use these data to develop models for customer utility
in different gaming platforms. The models provide data-driven
insight into potential causes for and against cloud gaming.


~\\
This paper is structured as follows: §~\ref{sec:relatedwork} provides a
brief overview of the related work. Afterwards, §~\ref{sec:background}
explains the necessary terms and technical details. The main part of
this work encompass §§~\ref{sec:engagement} which investigates
the consumer utility of \gls{cg} providers' service offerings
and business cases.
The paper concludes in §~\ref{sec:conclusion} with some
remarks and an outlook.

%and their competitions' business models and engagement from the angle of the user and the platform operator respectively.

%%%%%%%%%%%%
% \subsection{Fragestellungen}

% \begin{itemize}
% 	\item Kostenmodell für Cloud Gaming Provider?
% 	\item Attraktivität für “Core Gamer”?
% 	\item Wieviel ist eine NutzerIn bereit für einen Streaming Service mit einem bestimmten Spieleangebot und einer bestimmten Streaming- Qualität (Video-Qualität, Latenz, Grafikeinstellungen des Spiels) zu zahlen?
% 	\item Can they be competitive against other gaming platforms, both from provider as well as customer perspective
% 			Most challenging: can it beat the Steam price model and quality/number of games? (plus bundle providers and sales)
% \end{itemize}
% Netflix-Analogie?
