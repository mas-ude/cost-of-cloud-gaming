%!TEX root = paper.tex
%%%%%%%%%%%%%%%%%%%%%%%%%%%%%%%%%%%%%%%%%%%%%%%%%%%%%%%%%%%%%%%%%%%%%%%%
%%%%%%%

\section{Introduction}

\Gls{cg} has become quite a popular research topic in recent years.
Much of this research is aimed at comparing the experienced quality of
\gls{cg} to that of conventional gaming approaches, often with
reasonable results for the streaming approach (cf., for example,
\cite{5976180}). So, if there are only negligible quality drawbacks,
what about the commercial success of \gls{cg}? Intuitively, one
would assume that this could yield substantial benefits in terms of cost
and flexibility as a result of scaling effects through the used
\gls{cg} hardware in comparison to equipment-heavy classical home gaming
approaches. However, the \gls{cg} market seems to stagnate with a
high rate of fluctuation on the market, i.e., a constant stream of
market entrances and exits. For example, one of the most prominent
services in the past, \textsc{OnLive}, ceased to exist in 2015.

%shut down and was forced to sell off its remaining assets.

%\todo[inline]{PZ: ``sell off its remaining assets'' -- means?}
%\todo[inline]{FM: intellectual property, mainly the remaining patent portfolio (to Sony)}

%In theory, there could be large economic benefits for users that should foster a quick adaptation of such services. quite a few commercial services have already come and gone, with a rather high rate of fluctuation.

\Gls{cg} platforms have tested many different technical, service
and pricing models already. They currently adopt either
\textsc{Netflix}-like pure subscription models, surcharge for
additional games, or employ usage-based billing. Regardless, the public
interest remains low. This might be attributed to the broad range of
available, established substitutes, e.g., non-\gls{cg} platforms
such as video game consoles or PCs ---
with \steam\maybefootnote{\url{http://store.steampowered.com/}} as one of the
largest contenders. The move to digital distribution made gaming on PCs
quite popular, and PC games pricing became much more dynamic and
affordable in the process. Additionally, their offer is much larger
and more diverse.

%On the surface, current \gls{cg} services attempt to adopt a
%\textit{fixed fee subscription} model over the traditional \textit{à la
%carte} model. Fixed fee subscription proved to be hugely successful for
%other types of media, e.g., \textsc{Netflix} for movies and shows or
%\textsc{Spotify} for music. However, these two types of services offer a
%much larger catalogue of content at a comparable or even lower price
%than \gls{cg} services. Additionally, streaming asynchronous
%non-interactive media is technically less demanding (and thus cheaper to
%operate) than maintaining a quality level on par with that of locally
%running games.

The research question explored in this work considers the
utility offered to customers by current \gls{cg}
platforms, as compared to gaming platforms for the \acrshort{PC}.
Various utility metrics such as the number of offered games,
play-through lengths, review scores, and prices are investigated.
The evaluation is based on combining and investigating large
datasets of user choice and game properties.
Finally, a model is proposed that expresses customer utilityg
as a function of cost in different gaming platforms.
The se results provide data-driven
insight into potential causes for and against \gls{cg}.

This paper is structured as follows: §~\ref{sec:relatedwork} provides a
brief overview of the related work. §~\ref{sec:background}
explains the technical background, ecosystems, and utility metrics.
§§~\ref{sec:eval} and \ref{sec:utilitymodel} investigate the utility of
(cloud) gaming providers' service offerings and business cases,
and present a cost utility model for comparing the offers.
The analysis is based on a dataset spanning more than two years and
multiple platforms.
The paper concludes in §~\ref{sec:conclusion} with
%some remarks and
directions for future work.

%and their competitions' business models and engagement from the angle of the user and the platform operator respectively.

%%%%%%%%%%%%
% \subsection{Fragestellungen}

% \begin{itemize}
% 	\item Kostenmodell für Cloud Gaming Provider?
% 	\item Attraktivität für “Core Gamer”?
% 	\item Wieviel ist eine NutzerIn bereit für einen Streaming Service mit einem bestimmten Spieleangebot und einer bestimmten Streaming- Qualität (Video-Qualität, Latenz, Grafikeinstellungen des Spiels) zu zahlen?
% 	\item Can they be competitive against other gaming platforms, both from provider as well as customer perspective
% 			Most challenging: can it beat the Steam price model and quality/number of games? (plus bundle providers and sales)
% \end{itemize}
% Netflix-Analogie?
