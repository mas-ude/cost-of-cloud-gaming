%!TEX root = paper.tex
%%%%%%%%%%%%%%%%%%%%%%%%%%%%%%%%%%%%%%%%%%%%%%%%%%%%%%%%%%%%%%%%%%%%%%%%%%%%%%%

\todo[inline,color=red!10]{Other title could be: The Prospects of Cloud Gaming: Do ... }

\section{Introduction}

Cloud gaming has become quite a popular term in recent years in research. Much of this research is aimed at comparing the experienced quality of cloud gaming to that of conventional gaming approaches with the results generally in favor of conventional gaming, but only by a tiny margin. Some publications even praise the method and attest high \gls{QoE} values.

So, if there are only negligible quality drawbacks --- according to the literature --- what about the commercial success of cloud gaming? In theory, there could be large economic benefits for users that should foster a quick adaptation of such services. But looking back to reality, quite a few commercial services have already come and gone, with a rather high rate of fluctuation. For example, one of the most prominent services in the past, OnLive, shut down and was forced to sell off its remaining assets. All of these services, current as well as past, come with some subtle differences in their service, including the technical aspects, the selection of games, as well as the pricing model.

However, none of these seems to have garnered much public interest. Some of this can be attributed to a circumstance that is often overlooked, when evaluating cloud gaming services: the strong competition with other non-cloud gaming platforms, such as the individual console platforms or the large PC market --- with the Steam\footnote{\url{http://store.steampowered.com/}} platform as on of its strongest contenders. Through the move to digital distribution the PC platform has become quite popular and PC games pricing has become much more dynamic and affordable in the process.

On the surface, current Cloud Gaming services attempt to a adopt a fixed fee subscription model over the traditional à la carte model. This move proved to be hugely successful for other types of media, be it for example \textit{Netflix} for movies and shows or \textit{Spotify} for music. However, these two types of services seem to offer much more content at a comparable or even lower price point than Cloud Gaming services do. Additionally, the technical requirements to maintain a quality level on par with that of locally run games are also rather demanding.

The two main questions that this work aims to tackle are thus: \textit{``Can Cloud Gaming be attractive for users in today's highly competitive market?''} and \textit{``Can you operate a Cloud Gaming service with acceptable margins while maintaining acceptable quality levels?''}

Both questions are strongly intertwined. [...]

In practice one could actually simplify both questions quite easily to one: \textit{``Can you compete with the PC and Steam ecosystem (in terms of quality, prices, variety)?''}

In order to answer these questions, this paper looks at the perspective of users and service provider in separate and provides arguments backed by data and simple models. The initial insights gained from these do not shed a good light on the commercial future of Cloud Gaming services in general.

[...]
~\\
This paper is structured as follows. [...] 
Related work in Sec.~\ref{sec:relatedwork}
Background, technical details in Sec.~\ref{sec:background}
Customer side model and engagement metrics in Sec.~\ref{sec:engagement}
Service provider ``cost'' models in Sec.~\ref{sec:suppliermodelling}
Concluding remarks and outlook in Sec.~\ref{sec:conclusion}


%%%%%%%%%%%%
% \subsection{Fragestellungen}

% \begin{itemize}
% 	\item Kostenmodell für Cloud Gaming Provider?
% 	\item Attraktivität für “Core Gamer”?
% 	\item Wieviel ist eine NutzerIn bereit für einen Streaming Service mit einem bestimmten Spieleangebot und einer bestimmten Streaming- Qualität (Video-Qualität, Latenz, Grafikeinstellungen des Spiels) zu zahlen?
% 	\item Can they be competitive against other gaming platforms, both from provider as well as customer perspective
% 			Most challenging: can it beat the Steam price model and quality/number of games? (plus bundle providers and sales)
% \end{itemize}
% Netflix-Analogie? 