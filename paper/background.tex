%!TEX root = paper.tex
%%%%%%%%%%%%%%%%%%%%%%%%%%%%%%%%%%%%%%%%%%%%%%%%%%%%%%%%%%%%%%%%%%%%%%%%%%%%%%%%
\section{Background}
\label{sec:background}

This section reviews the current hardware and software platforms
for different types of gaming, introduces utility metrics for games,
and describes the data sources used for the evaluations in this paper.
All costs are from an European, specifically German, perspective. If a
product is not available in this region, the prices are converted using
the most recent currency exchange rates.
Note that as mentioned before, browser and mobile games are out
of scope for this work.

%!TEX root = paper.tex

\begin{table*}
\centering
\caption{Content and price models of cloud gaming services and select digital distribution platforms. If not stated otherwise, service is described from the EU/Germany region}
\label{tab:game-services}
	\begin{tabu}{X|X[r]X[r]X[r]X[r]X[r]}
	\toprule
	Service & Price Model & Model & No. of Titles & Regional Restrictions  & Notes\\
	\midrule
	\textsc{GeForce Now} for PC &  time-based \$25 for 20h resp. 10h & bring-your-own-PC-games & n/a & NA & service currently in free beta, prices based on latest announcement\footnote{\url{https://arstechnica.com/gaming/2017/01/nvidia-geforce-now-streaming-pc-mac-price-release-date/}}\\

	\textsc{GeForce Now} & base fee, 9.99€ monthly, some games incur additional charge & curated PC games & 55 included, 63 surcharge & NA, EU, JP & Only available on SHIELD devices \\ %device cost >=200€, price in germany, lower in US


	\psnow & flat subscription, 16,99€ monthly & select PS3/PS4 games & $432$ & \SI{12.26}{\hour}\\

	GameFly & ? & ? & ? & ? & Only for specific set-top boxes\\
	LiquidSky & time-based + subscription, 19.99\$ for the first 80h per month & bring-your-own-PC-games & n/a & NA, EU, Hong Kong & Limited storage\\


	\steam & direct sales, 3rd party vendors and bundles & open publication (100\$ fee) & $17111$ & pricing and availability differing across regions & \\


	\midrule
	Defunct Services & & & & & \\

	OnLive & & & & & Bankrupt, most patents sold to Sony (2015) [Source?]\\
	Gaikai & n/a & Streaming of short game demos in the Web browser & n/a & n/a & Bought by Sony (2012) [Source?]\\
	Gface (CryTek) & & & & &  Did not leave beta (2012), now just a company game launcher (also defunct)[Source?]\\
	Playcast & unknown & unknown & 50 (est.) & Only select ISP installations (Israel, Portugal, South Korea, France)& Only for specific set-top boxes, Merged with gamefly (2015) [Source?]\\

	\bottomrule
	\end{tabu}
\end{table*}

% not considered here, because not cloud gaming according to our definition
% but game streaming in the sense of: downloaded-as-you-play-but played-locally
% this 
% Utomik & flat  & ? & ? & ? & Only for specific set-top boxes\\





% NA: north america, EU: Europe, JP: Japan
% psn games from https://www.playstation.com/de-de/explore/playstation-now/ps-now-games/
% we should also refer to https://en.wikipedia.org/wiki/Cloud_gaming#Cloud_gaming_services on which this table is partially based

%%%%%%%%%%%%%%%%%%%%%%%%%%%%%%%%%%%%%%%%%%%%%%%%%%%%%%%%%%%%%%%%%%%%%%%%%%%%%%%%
\subsection{Gaming Hardware}

\subsubsection{Gaming \acrshort{PC}s}
Hardware viable for \gls{PC} gaming starts at about $\text{\texteuro} 500$ but
has practically no upper limit for enthusiasts. The \gls{GPU}
is a cost driver, and it is essential for modern \gls{PC} gaming.
This poses a certain financial barrier for customers to start \gls{PC} gaming,
which is however compensated by an increased flexibility and longevity of
hardware.

\subsubsection{Video Game Consoles}
Dedicated consoles represent the classical approach to video gaming.
The price for (non-portable) consoles varies but usually lies between
$\text{\texteuro} 300$ and $\text{\texteuro} 400$ for the latest console
generation, i.e., \textit{Switch}, \textit{PlayStation 4}, and
\textit{Xbox One}.
While priced lower than \glspl{PC}, console hardware is not componentized
and not upgradeable as easily.

\subsubsection{Cloud Gaming Hardware}
One of the main claimed benefits of \gls{cg} is its low requirements
on end user hardware. Thus, a less powerful \gls{PC} may suffice to
use a \gls{cg} service as long as the network connection to the
service provides an adequate latency and bandwidth.
Sony's \psnow \gls{cg} service was originally available only for their
PlayStation 3 and 4 consoles, and a line of their TV sets. The latter
option has since been removed, and streaming to \glspl{PC} enabled
instead.
% https://www.playstation.com/en-gb/explore/playstation-now/ps-now-on-pc/

Companies also experiment with devices that focus exclusively on streaming
\gls{cg} content. The only standalone game streaming device currently
in the market is NVIDIA's \textit{SHIELD} which starts at
$\text{\texteuro} 230$ and is bound to their \gls{cg} service, \gfnow.
% Price via Amazon.de
Here, too, a stream-to-\gls{PC} option has become available recently
(as a US-only beta programme).
% https://www.nvidia.com/en-us/geforce/products/geforce-now/mac-pc/



\subsection{Gaming Ecosystems}
Below, currently active (cloud and non-cloud) gaming platforms are examined
with regards to pricing models and hardware requirements and costs. The
information presented was collected between July 2015 and October 2017.

\subsubsection{\gls{PC} Gaming: Something for everyone}
\label{sec:pcgaming}

The rise of easy-to-use digital distribution platforms and the
independent (``indie'') game scene reinvigorated \gls{PC} gaming just a few
years ago. Today, \gls{PC} gaming is dominated by large digital marketplaces,
with \steam being the largest. The platform has about $10$ million
concurrent users at most times of the day. It periodically offers large,
often seasonal, sales of recent games at greatly reduced prices (rebates
of 75\% for a year-old game are not uncommon). In addition, many
resellers offer digital codes for other platforms, often at much lower
prices.
% Alternative storefronts, entirely independent from \steam, also exist, such as \textit{GOG}\footnote{\url{https://www.gog.com/}}.%, and enrich the competition even more.
Major releases on PC are usually priced between $\text{\texteuro} 50$
and $\text{\texteuro} 60$. However, due to the competition between the
vendors, the digital retail prices are significantly lower even at
launch, and also drop more quickly. Another recent trend are game
bundles, which especially prevalent in the indie games scene, commonly
offered with a pay-what-you-want model. \textit{Humble
Bundle}\footnote{\url{https://www.humblebundle.com/}} is a prominent
example.
% Number of games >>> what consoles have

%recurring bundles/subscriptions (humble monthly)
%	console/netflix similarities

% Steam Sales:
% Large seasonal sales (christmas, summer, lunar new year, halloween, fall, spring, ...) of many/most games on the platform usually rebates of 50\% and up.
% Weekly sales
% Daily sales
% Weekend sales
% Free weekends




\subsubsection{Video Game Consoles}

New, major game releases are mostly priced at either
$\text{\texteuro} 60$ or $\text{\texteuro} 70$. Once on the market, the
game prices decrease rather slowly. In recent years, retail stores have
been complemented with console-specific, proprietary digital
distribution services that also offer the latest game at the full price.
These official stores are usually exclusive vendors for digital game
codes where competitors are excluded.

%, meaning that there will be no competition that quickly reduces prices.
Subscription fees often apply for the multiplayer mode of games, e.g.,
\textit{PlayStation Plus} or \textit{Xbox Live Gold} with annual prices
of $\text{\texteuro} 50$ and $\text{\texteuro} 60$, respectively. These
services also include access to a small, monthly changing palette of
older titles.

% generally offer a monthly rotating palette of (older or smaller) game titles included in the subscription.



\subsubsection{\Gls{cg} Ecosystems}

\todo[inline]{Warum beschreiben wir genau die?}

NVIDIA's \gls{cg} platform%\footnote{\url{https://shield.nvidia.com/games/geforce-now}}
\gfnow
is available in North America and select European countries.
In Germany the service currently offers $68$ PC titles
for a monthly subscription fee of $\text{\texteuro} 10$. An additional
per-game one-time fee between $\text{\texteuro} 13$ and
$\text{\texteuro} 60$ is charged for the access to the $19$ most
prominent and recent games. The service is delivered from six
specialized data center locations (Dublin and Frankfurt in Europe).

The requirements to use this service are rather steep, demanding
\SI{50}{\mega\bit\per\second} for a full
1080p60 stream\footnote{\label{foot:rate}Please note: This frame rate of
\SI{60}{\hertz} represents the rate of the video encoder and not the
game's actual frame rate, which might be considerably lower depending on
the complexity of the game.} (\SI{10}{\mega\bit\per\second} in
order to use the service at all) and a maximum \acrshort{RTT} of
\SI{60}{\milli\second} to one of the data centers. In addition,
streaming is exclusive to \textit{SHIELD} devices which start at
$\text{\texteuro} 200$.

% Started in parts of Europe in Q4/2015

\psnow, Sony's cloud gaming service, offers to stream titles from previous
PlayStation generations, as the latest console generation lacks
backwards compatibility. It is currently available in North America and
the UK, with a closed beta running in other European countries and
Japan. The offered titles and exact pricing vary from country to
country. For the UK, about $190$ titles are available, and most titles
are covered by the monthly subscription fee of about $\text{\texteuro}
17$. All titles are also available through a separate rental service,
costing about $\text{\texteuro} 4$ for \SI{48}{\hour} and
$\text{\texteuro} 10$ for one month of access. This is in addition for
the device cost, as the service is only available on PlayStation 4 and 3
consoles as well as some select Sony TVs and other devices with extra
game controller.

%(which would however necessitate the purchase of a game controller).

The streaming itself is performed at a resolution of
720p60\cref{foot:rate} requiring a \SI{5}{\mega\bit\per\second}
connection. Reports on the video quality have been rather
mixed.\footnote{\url{http://www.eurogamer.net/articles/digitalfoundry-2015-hands-on-with-playstation-now}}
% Sony is also reportedly using specialized server hardware, adapted from regular PlayStation 3s, effectively eliminating any chance for multi-purpose uses of these devices. This could lower scalability gains.

% cost and some more infos:
% http://www.pocket-lint.com/news/126394-playstation-now-subscription-service-comes-to-the-uk-what-is-it-and-how-can-you-get-it

% (US/UK only?) Deuschlandbeta seit ~Q4/2015

\todo[inline]{\liquid ist auch sehr schön!}



\subsection{Utility Metrics}

Utility metrics help to estimate the value of a service for a customer.
It should be noted that no single metric is likely of value by itself,
yet all metrics somehow influence customer decisions.

The first basic metric is the number of games offered. A larger number
means more choice for the customer, but ``spam'' phenomena need to
be considered, as large offers might include many games of sub-par quality.

Second, the number of players already owning a game can be a
motivating factor for other customers to choose the same game.
One may interpret this as a form of preferential attachment.
This is particularly interesting for decidedly online games, i.e.
such where players interact in the game environment over the network,
so that more interactions are possible if more players are online.

Then, games are priced differently, both within and across
platforms. Many games are offered completely free of charge.
Some are funded indirectly through in-game purchases. On the
other end of the spectrum, titles whose development was expensive
and for which the audience anticipation is high will probably sell
for a correspondingly high price.

Furthermore, game lengths come to mind. Games range from very short
(on the order
of minutes) to tens of hours (for games with an extensive storyline),
but need not be limited at all (e.g. explorative or ``open world'' games).
Game length should thus be considered differently depending on
the player type. For casual gamers, shorter games probably have
more appeal, as their limited playtime is more likely to allow
players to complete them. ``Hardcore'' gamers that spend more
time gaming overall might find longer games more engaging.
Overall, neither mode intrinsically offers higher utility.

Next, the age of games (per their original release date) is of
interest. Some players favor the newest games, others prefer ``classics'';
any focus or combination thereof will likely attract specific
proportions of customers. Also, newer games tend to also be topics
in the media (and in advertising, increasing their publicity),
offer a greater diversity (e.g., regarding game mechanics and control),
% Wii once was new!
and feature improved technical quality (resolution, scene complexity).


There also exist considerable amounts of secondary literature about
games in the form of reviews. These can be viewed as social factor,
in that gamers might be drawn to positively-reviewed titles more
strongly, or rather stay away from less favorably reviewed ones.
Some media aggregate reviews from many outlets and
are thus of particular interest for an analysis.

Many other utility metrics are imaginable, e.g.
the number of platform ``exclusive'' game titles,
the game genre and other classifications,
the number of game sales and subscriber numbers,
technical aspects like graphical fidelity, performance, precision
and responsiveness of controls,
measures of the game's content like variety and quality of game mechanics,
or other content-centric factors.

Lastly, any combination of the metrics mentioned above has its
merits for assessing the utility of a gaming platform. For example,
the combined expected playthrough length of the top 10\% of games
across platforms could be evaluated, and judged against the amount
of money required to equip each platform like this.


\subsection{Data Sources}
In order to investigate the utility metrics described, data was collected from
multiple sources and merged (with a few instances of omission and
double-count) into a data base. In the interest of repeatability and
reproducibility, all of the data reported on in this work, as well as
the code used to collect and process it, can be found in public
repositories\footnote{The main repository can be found at
\url{https://github.com/mas-ude/cost-of-cloud-gaming}}.
Data were joined on game names and platforms (where appropriate).
The different data sources are described in the following.

For \steam, the public \acrshort{REST} \acrshort{API} was used to
fetch the name and current price of each game at different points in
time\footnote{\url{https://github.com/mas-ude/steam-data-stats}}. This
data was combined with \acrshort{API} data from the 3rd-party site
\textit{SteamSpy}\footnote{\url{https://steamspy.com}}, which parses all
publicly visible \steam user profiles and
estimates statistics on the size of the player base and the time each
player spends with a title. \textit{SteamSpy} also provides a heuristic
projection of the total number of owners of each listed title on \steam.
For \gfnow, game names and prices were screen-scraped manually.
Game names for \psnow is available from their website\footnote{\url{https://www.playstation.com/de-de/explore/playstation-now/ps-now-games/}}.
Through this, the portfolios can be compared between platforms.

Game publishers seldom publish intended playthrough
length of games, and not all games necessarily have one. However, players
may self-report their experienced playthrough times on sites like
\hltb\footnote{\url{http://howlongtobeat.com/}}, which was web-scraped
for this analysis using a custom tool\footnote{\url{https://github.com/mas-ude/gamelengths-scraper}}.

%Reported times are separated into different play styles, e.g., ``main story'' for players that just finish the main objectives of a game, or ``completionist'' for those who explore the game to the fullest extent. Only an aggregate time over all reports per play style is shown for every game. (For our analysis, we average over all play styles for every game.) %It should however be stressed again, that this is a self-reporting site without strong validity checks, which has to be considered when contemplating the accuracy of the data.

Lastly, the age of a game is computable from its release date. To this end, the
\metacritic\footnote{\url{http://www.metacritic.com/}} page which
aggregates reviews of video games (and other media) was
scraped\footnote{\url{https://github.com/mas-ude/metacritic_scraper}}.

