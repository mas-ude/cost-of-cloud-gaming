%!TEX root = paper.tex
%%%%%%%%%%%%%%%%%%%%%%%%%%%%%%%%%%%%%%%%%%%%%%%%%%%%%%%%%%%%%%%%%%%%%%%%%%%%%%%%
\section{Background}
\label{sec:background}

This section aims to cover all basic information on Cloud Gaming and their business models. It also briefly introduces the employed datasets.

%%%%%%%%%%%%%%%%%%%%%%%%%%%%%%%%%%%%%%%%%%%%%%%%%%%%%%%%%%%%%%%%%%%%%%%%%%%%%%%%
\subsection{Service Costs of Gaming Platforms}

To evaluate the costs and benefits of individual gaming platforms, one should first take a look at the vastly different price models --- as well as the supporting competitive ecosystem surrounding them --- these services offer to their customers. For this investigation only currently active services, Cloud and otherwise, are examined which narrows the number of services considerable. The information presented here was collected in February 2016 and is subject to change. All costs shown here are from an European, specifically German, perspective. If a product is not available in this region, the prices are converted accordingly to the current exchange rates.


\paragraph{Video Game Consoles}

One of the classical approaches to video games is using dedicated consoles with physical copies of a game bought at a retailer. The price for (non-portable) consoles varies but usually lies between \SI{300}[\EUR] and \SI{400}[\EUR] for the latest console generation, i.e. currently \textit{Wii U}, \textit{PlayStation 4}, and \textit{Xbox One}. New, major game releases are mostly priced at either \SI{60}[\EUR] or \SI{70}[\EUR]. The decline in price is rather slow for individual console games, often only happening over the course of several years.

In recent years retail stores have been more and more complemented with each console's proprietary digital distribution service, also the latest game at the full price. These official stores are usually the only vendor for digital game codes, meaning that there will be no competition that quickly reduces prices.

In order to utilize a game's multiplayer component, some services require you to opt in to a subscription, dubbed, e.g. \textit{PlayStation Plus} or \textit{Xbox Live Gold}, with annual prices of \SI{50}[\EUR] and \SI{60}[\EUR] respectively. These services do however generally offer a monthly rotating palette of (older or smaller) game titles included in the subscription.

% \paragraph{Historical Example: OnLive}


\paragraph{The PC Gaming Ecosystem}

The rise of easy-to-use digital distribution platforms and the independent (i.e. ``indie'') game scene a few years back reinvigorated PC gaming. Today, PC gaming is dominated by large digital marketplaces, and Steam in particular. The platform has fully embraced its digital and online-only nature and periodically offers large sales of recent games at greatly reduced prices (rebates of 75\% for a year-old game are not uncommon). In addition, many resellers offer digital codes for other platforms, often at much lower prices. Alternative storefronts, entirely independent of Steam, also exist, such as \textit{GOG}\footnote{\url{https://www.gog.com/}}, and enrich the competition even more.

Another recent trend are game bundles, especially prevalent in indie games, which bundle together a number of games for either a low price or, more commonly, with a pay-what-you-want model with parts of the money going to charities. \textit{Humble Bundle}\footnote{\url{https://www.humblebundle.com/}} is a prominent example for this. The official prices for new major releases on PC are comparable to that of console titles. However, due to the competition the street prices are significantly lower even at launch, and also drop more quickly.

The barrier to entry to PC gaming is a bit steeper than for consoles, albeit with increased flexibility and longevity. Hardware viable for PC gaming starts at about \SI{500}[\EUR] but has practically no upper limit for enthusiasts. The main spender will usually be the \gls{GPU}, often surpassing the \acrshort{CPU} in its performance impact in today's games by far.

%recurring bundles/subscriptions (humble monthly)
%	console/netflix similarities


\paragraph{Geforce Now}

\textit{Geforce NOW}\footnote{\url{http://shield.nvidia.com/game-streaming-with-geforce-now}} is a Cloud Gaming service by NVIDIA available in North America as well as some European countries. In Germany the service currently offers access to $68$ PC titles for a monthly subscription fee of \SI{10}[\EUR]. The more prominent and recent $19$ of those titles however require an additional one-time payment between \SI{13}[\EUR] and \SI{60}[\EUR]. The service is delivered from six specialized data center locations, with Dublin and Frankfurt responsible for Europe. 

The requirements to use this service are rather steep, demanding \SI{50}{\mega\bit\per\second} for a full 1080p60 stream (\SI{10}{\mega\bit\per\second} in order to use the service at all) and a maximum \acrshort{RTT} of \SI{60}{\milli\second} to one of the data centers. In addition, streaming is exclusive to one of NVIDIA's \textit{SHIELD} devices which start at \SI{200}[\EUR].

% Started in parts of Europe in Q4/2015


\paragraph{Playstation Now}

In the absence of backwards compatibility in the latest PlayStation generation, the \textit{PlayStation Now} service was put in place to stream titles from the PlayStation 2 and 3 generation to the current iteration of Sony's consoles. It is currently available in North America and the UK, with a closed beta running in some other European countries as well as Japan.

Both the offered titles as well as the precise pricing vary from country to country. As stated this work will focus on the European perspective. For the UK, about $150$ titles are available. A large portion of them is included in the subscription service, incurring a monthly fee of about \SI{17}[\EUR]. All titles are also available through a separate rental service, costing about \SI{4}[\EUR] for \SI{48}{\hour} of access, and \SI{10}[\EUR] for a week. This is in addition for the device cost, as the service is only available to the PlayStation 4 and 3 consoles as well as some select Sony TVs and other devices (which would however necessitate the purchase of a game controller).

The streaming itself is performed at a resolution of 720p60 requiring a \SI{5}{\mega\bit\per\second} connection. Reports on the video quality have been rather mixed.\footnote{\url{http://www.eurogamer.net/articles/digitalfoundry-2015-hands-on-with-playstation-now}}

% cost and some more infos:
% http://www.pocket-lint.com/news/126394-playstation-now-subscription-service-comes-to-the-uk-what-is-it-and-how-can-you-get-it

% (US/UK only?) Deuschlandbeta seit ~Q4/2015


%%%%%%%%%%%%%%%%%%%%%%%%%%%%%%%%%%%%%%%%%%%%%%%%%%%%%%%%%%%%%%%%%%%%%%%%%%%%%%%%
\subsection{Platform Provider Cost Factors}
Backend/Service Requirements and Demands

%%%%%%%%%%%%
\subsubsection{CAPEX}

\begin{itemize}
	\item Regionale Data Center
	\item Gaming Server (GPU-Enabled)
	\item Entwicklungskosten für Software-Plattform(?)
\end{itemize}

\paragraph{Hardware}

\url{https://www.nvidia.com/object/cloud-gaming-gpu-boards.html}
\url{https://www.nvidia.com/object/grid-technology.html}


%%%%%%%%%%%%
\subsubsection{OPEX}

\paragraph{Verkehrsvolumen}

\begin{itemize}
	\item Internetanbindung?
	\item Caching of basic resources is probably not applicable?
\end{itemize}

\paragraph{Serverlaufzeiten}

\begin{itemize}
	\item Energie
	\item Verschleiß
	\item Wartungs- und Betriebspersonal oder Anmietung
	\item Frage: Rechnet sich Anmietung von Ressourcen bei großen generischen Rechenzentren? Annahme nein, da man selbst ein großer Anbieter wäre u. die Margin wegfallen. Auf der anderen Seite gibt es Hardware die für Games im Serverbereich besser skalieren? Wenn ja, kann umso mehr kein generischer Anbieter die Lösung sein
\end{itemize}

\paragraph{Spiele-Lizenzen und -Adaptionskosten (?)}
Modelannahme: Kosten pro Nutzung (realistisch eher in Blöcken verrechnet)





%%%%%%%%%%%%%%%%%%%%%%%%%%%%%%%%%%%%%%%%%%%%%%%%%%%%%%%%%%%%%%%%%%%%%%%%%%%%%%%%
\subsection{Data}

The leading question for the user side of the cost-benefit analysis is: ``How can we properly estimate the value a user gets from a specific service?'' with an additional question of ``What is the definition of value in this context?''

Simply counting the number of games one gets for a specific amount of money may be the easiest value metric but may also fall short. Therefore, this paper evaluates further approaches that also take game lengths and press review scores into account.

In order to achieve this, data had to be collected from various sources and merged into a consistent data base. All the scraping and merging code as well as the data itself can be found in the repositories at TODO to be verified by third parties. The following sections describe the individual data sources.

%%%%%%%%%%%%
\subsubsection{Steam API + SteamSpy Datensätze+Graphen}

Steam Sales:
Large seasonal sales (christmas, summer, lunar new year, halloween, fall, spring, ...) of many/most games on the platform usually rebates of 50\% and up.
Weekly sales
Daily sales
Weekend sales
Free weekends



\url{https://github.com/mas-ude/steam-data-stats} Steam + SteamSpy REST API Datensammler + ein paar Graphplotter (und eher ergebnislose Clustering-Versuche); für sinnvolle Analsyen müsste man dazu aber viel häufiger Datensätze generieren. Beispielausgaben:

CDF der Preise auf Steam (Juli ‘15) \ref{fig:steam-prices}

\begin{figure}[!t]
	\centering
	\includegraphics[width=1.0\columnwidth]{images/steam-prices.pdf}
	\caption{CDF of games on the steam platform at two distinct dates.}
\label{fig:steam-prices}
\end{figure}

Violinenplot der durchschnittlichen Spielzeit aufgeteilt auf unterschiedliche Preiskategorien. \ref{fig:steam-cost-vs-playtime-violin}

\begin{figure}[!t]
	\centering
	\includegraphics[width=1.0\columnwidth]{images/steam-cost-vs-playtime.pdf}
	\caption{Violin plot of the average playtime (as recorded by SteamSpy) of games categorized by their prices.}
\label{fig:steam-cost-vs-playtime-violin}
\end{figure}

%%%%%%%%%%%%
\subsubsection{HowLongToBeat Data}

Data scraped from \url{howlongtobeat.com}. This site allows for manual reporting of playthrough times of games on any platform. Times are separated into different play styles (e.g. ``main story'', ``completionist'') and only an aggregated time is shown. For this analysis only the average playtime of all play styles is taken into account.
It should however be stressed again, that this is a self-reporting site without strong validity checks. This has to be considered when contemplating the accuracy and validity of the data.

\begin{figure}[!t]
	\centering
	\includegraphics[width=1.0\columnwidth]{images/gamelengths-density.pdf}
	\caption{Density plot of the average game lengths over all play styles.}
\label{fig:gamelengths-density}
\end{figure}

Nonetheless, the distribution of game lengths from this set can still be worth to look at in the context of putting value to games for consumers of different platforms. As seen in Figure~\ref{fig:gamelengths-density} the play lengths vary greatly, with the median at \SI{7.5}{\hour} and a long tail of long play times reaching \SI{420}{\hour}.


%%%%%%%%%%%%
\subsubsection{Metacritic Data}

Game lengths can not only serve as an indicator of the amount of content a game has to offer, but can also serve as an engagement metric to estimate a user's amount of satisfaction. More fitting engagement metrics could also be employed.

% TODO: include or compare with data from opencritic.com as soon as their API is public/usable

\begin{figure}[!t]
	\centering
	\includegraphics[width=1.0\columnwidth]{images/releases-per-year.pdf}
	\caption{Number of game releases per platform according to the Metacritic data.}
\label{fig:releases-per-year}
\end{figure}




