%!TEX root = paper.tex
%%%%%%%%%%%%%%%%%%%%%%%%%%%%%%%%%%%%%%%%%%%%%%%%%%%%%%%%%%%%%%%%%%%%%%%%%%%%%%%



\section{Fragestellungen}

\begin{itemize}
	\item Kostenmodell für Cloud Gaming Provider?
	\item Attraktivität für “Core Gamer”?
	\item Wieviel ist eine NutzerIn bereit für einen Streaming Service mit einem bestimmten Spieleangebot und einer bestimmten Streaming- Qualität (Video-Qualität, Latenz, Grafikeinstellungen des Spiels) zu zahlen?
\end{itemize}

Netflix-Analogie? 


\section{Kostenfaktoren}

\subsection{Cloud Gaming Provider}

\subsubsection{CAPEX}

\begin{itemize}
	\item Regionale Data Center
	\item Gaming Server (GPU-Enabled)
	\item Entwicklungskosten für Software-Plattform(?)
\end{itemize}

\subsubsection{OPEX}

\paragraph{Verkehrsvolumen}

\begin{itemize}
	\item Internetanbindung?
	\item Caching of basic resources is probably not applicable?
\end{itemize}

\paragraph{Serverlaufzeiten}

\begin{itemize}
	\item Energie
	\item Verschleiß
	\item Wartungs- und Betriebspersonal oder Anmietung
	\item Frage: Rechnet sich Anmietung von Ressourcen bei großen generischen Rechenzentren? Annahme nein, da man selbst ein großer Anbieter wäre u. die Margin wegfallen. Auf der anderen Seite gibt es Hardware die für Games im Serverbereich besser skalieren? Wenn ja, kann umso mehr kein generischer Anbieter die Lösung sein
\end{itemize}

\paragraph{Spiele-Lizenzen und -Adaptionskosten (?)}
Modelannahme: Kosten pro Nutzung (realistisch eher in Blöcken verrechnet)




\section{Daten}

Bereits Gesammelte Daten

\subsection{Steam API + SteamSpy Datensätze+Graphen}
\url{https://github.com/mas-ude/steam-data-stats} Steam + SteamSpy REST API Datensammler + ein paar Graphplotter (und eher ergebnislose Clustering-Versuche); für sinnvolle Analsyen müsste man dazu aber viel häufiger Datensätze generieren. Beispielausgaben:

CDF der Preise auf Steam (Juli ‘15) \ref{fig:steam-prices}

\begin{figure}[!t]
	\centering
	\includegraphics[width=1.0\columnwidth]{images/steam-prices.pdf}
	\caption{CDF of games on the steam platform at two distinct dates.}
\label{fig:steam-prices}
\end{figure}

Violinenplot der durchschnittlichen Spielzeit aufgeteilt auf unterschiedliche Preiskategorien. \ref{fig:steam-cost-vs-playtime-violin}

\begin{figure}[!t]
	\centering
	\includegraphics[width=1.0\columnwidth]{images/steam-cost-vs-playtime.pdf}
	\caption{Violin plot of the average playtime (as recorded by SteamSpy) of games categorized by their prices.}
\label{fig:steam-cost-vs-playtime-violin}
\end{figure}



\subsection{platform-market-comparison/games-per-year.R}

 hat den ersten Versuch einer Nutzenrechnung für Spieler auf verschiedenen Plattformen. Script könnte leicht angepasst und erweitert werden. Beispielausgabe \ref{fig:gamesperyear-over-budget}, \ref{fig:steam-prices}

\begin{figure}[!t]
	\centering
	\includegraphics[width=1.0\columnwidth]{images/gamesperyear-over-budget.pdf}
	\caption{Models for several platforms showing the number of games per year that can be bought with a specific \$ budget.}
\label{fig:gamesperyear-over-budget}
\end{figure}

\begin{figure}[!t]
	\centering
	\includegraphics[width=1.0\columnwidth]{images/games-over-year.pdf}
	\caption{Models for several platforms showing the number of games that can be bought over the years subscribed to / using this service.}
\label{fig:games-over-years}
\end{figure}
