%!TEX root = paper.tex
%%%%%%%%%%%%%%%%%%%%%%%%%%%%%%%%%%%%%%%%%%%%%%%%%%%%%%%%%%%%%%%%%%%%%%%%%%%%%%%
%\section{Cloud Gaming Provider Models}

\section{Supply-Side Efficiency Modelling} % OR ONLY: THE SUPPLIER'S PROBLEM
\label{sec:suppliermodelling}



%%%%%%%%%%%%%%%%%%%%%%%%%%%%%%%%%%%%%%%%%%%%%%%%%%%%%%%%%%%%%%%%%%%%%%%%%%%%%%%%
%\subsection{Platform Provider Cost Factors}
%Backend/Service Requirements and Demands

%%%%%%%%%%%%
%\subsubsection{CAPEX}

%\begin{itemize}
%	\item Regionale Data Center
%	\item Gaming Server (GPU-Enabled)
%	\item Entwicklungskosten für Software-Plattform(?)
%\end{itemize}

%\paragraph{Hardware}

%\url{https://www.nvidia.com/object/cloud-gaming-gpu-boards.html}
%\url{https://www.nvidia.com/object/grid-technology.html}


%%%%%%%%%%%%
%\subsubsection{OPEX}

%\paragraph{Verkehrsvolumen}

%\begin{itemize}
%	\item Internetanbindung?
%	\item Caching of basic resources is probably not applicable?
%\end{itemize}

%\paragraph{Serverlaufzeiten}

%\begin{itemize}
%	\item Energie
%	\item Verschleiß
%	\item Wartungs- und Betriebspersonal oder Anmietung
%	\item Frage: Rechnet sich Anmietung von Ressourcen bei großen generischen Rechenzentren? Annahme nein, da man selbst ein großer Anbieter wäre u. die Margin wegfallen. Auf der anderen Seite gibt es Hardware die für Games im Serverbereich besser skalieren? Wenn ja, kann umso mehr kein generischer Anbieter die Lösung sein
%\end{itemize}

%\paragraph{Spiele-Lizenzen und -Adaptionskosten (?)}
%Modelannahme: Kosten pro Nutzung (realistisch eher in Blöcken verrechnet)



%%%%%%%%%%%%%%%%%%%%%%%%%%%%%%%%%%%%%%%%%%%%%%%%%%%%%%%%%%%%%%%%%%%%%%%%%%%%%%%

%\subsection{Computational Efficiency}

This section assesses the supply-side of the cloud gaming market with a strong focus on costs. The remainder of this section first clarifies typical cost factors in the cloud gaming market and then continues with a specific efficiency model formulation that responds to the demand-side findings of previous sections.

\subsection{Cost Factors}

The dominating cost factor in cloud gaming is the server infrastructure, both in terms of \gls{CAPEX} and \gls{OPEX}. The low latency requirements of many games, renders the cloud gaming infrastructure more regionally-oriented, i.e., regional data centers are used to provide high-quality and low-latency service. Due to the specific demands of games, specialized hardware is used (e.g., with GPU-enabled CPUs), rather than generic cloud servers. These factors increase the \gls{CAPEX} for cloud service operators, but also lower the efficiency of the system, as more generic and more globalized cloud services cloud yield scaling advantages. Moreover, the rental of generic cloud service hardware or resources seems to be unrealistic due to the lowered efficiency --- cloud services are typically data-centric, while gaming is graphics-intense. The \gls{OPEX} is limited to maintenance activities, consumables, Internet access fees and energy.

In addition, game license fees have to be considered as noteworthy \gls{CAPEX} (or potentially also \gls{OPEX}). Depending on the licensing model, a flat license (one-time price for an unlimited number of subscribers), volume licenses, or per-use or per usage fee may be arranged with license owners. In all cases, it is realistic to assume that scaling effects exist: the initial \gls{CAPEX} is high, while marginal costs decrease for additional subscribers.

On the customer-side, the subscription or per-game prices represent costs, that create the revenue for the cloud gaming provider and for the game license owners. Customers typically also require specific gaming hardware to connect to the platform, which is cheaper than high-end gaming hardware, but may lower the added value over hardware-intensive conventional gaming approaches.

\subsection{Model}

Based on the collected consumer price figures, this section will elaborate on the required computational efficiency, i.e., cost per hosted subscriber, in order to successfully establish cloud gaming approaches on the market. Due to the limited available data, this investigation will follow a single data center assumption. Due to the demands of cloud gaming to serve both high performance and low latency, regional data centers will play a dominant role in the provider side cost modeling. Following this assumption, hereinafter a specific model is created that characterizes at which cost efficiency levels the cloud gaming business can be operated successfully.

\begin{figure}[!t]
	\centering
	\includegraphics[width=0.65\columnwidth]{images/overbooking_datacenter.pdf}
	\caption{Overbooking of available computational capacity.}
\label{fig:overbooking_datacenter}
\end{figure}

The computational efficiency considers the maximum overbooking rate $\epsilon \geq 1$, where $\epsilon = 1$ refers to no overbooking. Due to expected peaky demand curve, the overbooking rate $\epsilon$, as inverse of the capacity load $\rho$, is derived from the peak demand $L$ (simultaneously connected users) and the available data center capacity $Cap$ (number of users that can be handled),

\begin{align}
	\rho :=& \frac{\lambda}{\mu} = \frac{L}{Cap} \quad\text{,} \\
	\epsilon :=& \frac{1}{\rho} \quad\text{,}
	%\frac{Cap}{d_{peak}} \quad ,
	%\epsilon = \frac{1/rho} = 
\end{align}

where $\lambda$ is the average arrival rate of new users requesting to play a game (at peak times; e.g., per minute) and $\mu$ is the corresponding average handling of such requests by the server. This modeling stands in contrast to the classical formulation of Little's law, which focuses on long-term averages. The maximum number of subscribers $d$ (maximum service demand) is, thus, given by

\begin{equation}
	 d = Cap \cdot \epsilon \quad \text{.}
\end{equation}

The average monthly customer price $\bar{p}$ aggregates the monthly subscription fee and the customer's depreciation costs for the hardware investments on a four years investment duration. We further consider a minimum profit margin $m = \SI{3}{\percent}$, which is in line with the average figure for the global game industry\footnote{\url{http://www.polygon.com/2012/10/1/3439738/the-state-of-games-state-of-aaa}} and substantially below the cloud computing figures that can range up to \SI{16.9}{\percent}\footnote{\url{http://www.forbes.com/sites/georgeanders/2015/04/23/amazons-web-services-delight-16-9-margins-more-joy-ahead/\#73324aa64b4e}} and potentially even higher\footnote{\url{http://www.bloomberg.com/news/articles/2015-12-02/microsoft-should-disclose-cloud-revenue-margins-ballmer-says}}.

%Global Games statistics / billion revenues 2012-2016: http://newzoo.com/infographics/global-games-market-report-infographics-2013/
% Game industry = 3%: http://www.polygon.com/2012/10/1/3439738/the-state-of-games-state-of-aaa
% Game industry in the past (2009 – average console game with margin of 40%): http://www.businessinsider.com/casual-gaming-profit-margins-near-90-2009-10?IR=T
% Profit margins in cloud computing:
%	Amazon 16.9% (2015): http://www.forbes.com/sites/georgeanders/2015/04/23/amazons-web-services-delight-16-9-margins-more-joy-ahead/#73324aa64b4e
% 	Microsoft 44% (2015) -- questionable: http://www.bloomberg.com/news/articles/2015-12-02/microsoft-should-disclose-cloud-revenue-margins-ballmer-says

\begin{align} \label{eq:computational_efficiency}
	\frac{\epsilon \cdot Cap \cdot \bar{p}}{Cap} :=& \underbrace{\frac{\mathcal{C}_{cap}}{Cap}}_{\mathcal{C}_{u}} \cdot m\\
	%= C_{u} = \frac{C_{cap}}{Cap} :=& \frac{\epsilon \cdot \bar{p}}{m}
	\Longrightarrow \mathcal{C}_{u} :=& \frac{\epsilon \cdot \bar{p}}{m}
\end{align}

When treating the costs of the regional data center as a black box (operational and capital costs for the data center, and required game licensing fees), the analysis can concentrate on the required capacity and licensing cost $\mathcal{C}_{u}$ per user $u$ that can be connected at peak times. 

%connected user $u$ at peak .

%\begin{equation}
%	ce = \frac{C_{cap}}{Cap} \quad .
%\end{equation}

%%%%%%%%%%%%%%%%%%%%%%%%%%%%%%%%%%%%%%%%%%%%%%%%%%%%%%%%%%%%
%SOME DATA CONSIDERATIONS:


% ALTE DATEN:
% Anderer Messpunkt:
% Höchststand (simultaneous): 12 406 722 Nutzer maximal, Feb 13 - Feb 15
% Peak immer abends. Niedrigster Wert bei <7.5 Mio Nutzern
% => \epsilon von 75/12,406722 = 6,0451100621
% Eigentlich, da steam wächst, > 6 eine gute Annahme. Wir könnten versuchen Schranken zu definieren.
% Wenn wir annehmen, dass Skalierung gut funktioniert, benoetigen wir keinen Buffer. Sollen wir Buffer verwenden?

% NEUE DATEN:
% ANGLE 1:
%According to http://venturebeat.com/2014/01/15/steam-has-75-million-registered-users-third-party-steam-controllers-and-other-tidbits-from-valves-dev-days/
% Customer base of Steam was 75 Million active users in 2014. 
% %Vermutlich Nutzerzahl mittlerweile hoeher. Schaetzungen waeren also konservativ ausgerichtet.
%According to http://store.steampowered.com/stats/?l=german
% Testzeitraum: Feb 16 – 18
% Höchststand (simultaneous): 11 645 185	 Nutzer maximal
%	-> Anteil Grundgesamtheit: 0.1552691333
%	-> Epsilon: 6.4404301005 (75/11,645185)
% Tiefststand (simultaneous): 6 527 571 
%	-> Relatives Wachstum: 1.7839997451 = 78%

% ANGLE 2: 
% Data from ``Enabling Experiments for Energy-Efficient Data Center Networks on OpenFlow-based Platform''
% Data is on server loads for video traffic (Video on Demand) of a big ISP in Vietnam
% Does not give indications on parent population. So, we will compare the steam data with this data in a relativ fashion.
% Thursday values in Fig 2. are most peaky. So we take them.
% Höchststand: 71.190% at 17.530 (da time. Half past 5)
% Tiefststand: 7.971 % at 7.88 (day time time. Almost 8 o'clock – nobody watches video)
% 	-> Relatives Wachstum: 8.9311253293 = 893.1%
% Umgelegt vom Mindesstand bei Steam waere das dann:
%	6 527 571 * 8.9311253293 = 58.298.554,697 (geringfuegig verschoeben zu den initialen Daten%

%%%%%%%%%%%%%%%%%%%%%%%%%%%%%%%%%%%%%%%%%%%%%%%%%%%%%%%%%%%%

%\todo[inline]{NOW LET'S ADD THE DATA. Check the 12.4m. E.g. user older data. Introduce customer base data and reasoning above. Illustrate that epsilon will be around 6.}

When considering a measurement iteration between Feb 16 and 18, the minimum and maximum number of simultaneously connected users\footnote{\url{http://store.steampowered.com/stats/}, accessed: Feb 18, 2016} is $6.53\text{m}$ and $11.65\text{m}$ respectively. Setting the maximum in relationship to the $75$ million active steam users\footnote{http://venturebeat.com/2014/01/15/steam-has-75-million-registered-users-third-party-steam-controllers-and-other-tidbits-from-valves-dev-days/, last accessed: Feb 18, 2016}, we can calculate an $\epsilon_{\text{Steam}}$ of $6.44$. For cross-validation purposes, link load considerations for other media streaming services may also be considered: When comparing the relative load level change between the minimum and maximum utilization of a large Vietnamese \acrshort{ISP}'s \gls{VoD} streaming server as given in \cite{thanh2012enabling}% (reconstructed data for Thursday in Fig 2. as most peaky day)
, with the comparable change in the case, we would obtain a maximum number of simultaneously connected users of $58.3$ million and an $\epsilon_{\text{sub}}$ of $1.29$. While the discrepancy may appear high, the associated payment modalities may be at cause: While \steam sells game licenses, \gls{VoD} services often use a subscription services (a flat rate for a given time period). For the subsequent analysis we will consider the range between the minimum of $\epsilon_{\text{sub}}$ and maximum of $\epsilon_{\text{Steam}}$.

Moreover, the average price $\bar{p}$ is also parameterized on the basis of the collected \steam data. This data does, however, not reveal the actual $\bar{p}$ for a game, which also includes the historic price changes, but only the current price and the volume of sold games. In this respect, we can distinguish the average price for the current portfolio (portfolio price) and weighted according to the known volume (weighted price) for the assessment times of Summer 2015, end of October 2015 and March 2016 --- see Tbl.~\ref{tab:dataset-stats}. To reflect the range of the collected data, subsequently the minimum, i.e., $\bar{p}_{\text{min}} = 5.30$, and the maximum values, i.e., $\bar{p}_{\text{max}} = 12.39$, are used as inputs for the subsequent cost considerations.

\begin{table}
\centering
\caption{Average prices for steam games.}
\label{tab:dataset-stats}
\begin{tabu}{X[1.3]|X[r]X[r]X[r]}
	\toprule
	& \textbf{Summer 15´} & \textbf{End of Oct/15´} & \textbf{March 16´} \\
	\midrule
	\textbf{Portfolio price} & \SI{10.11}{\EUR}	& \SI{8.47}{\EUR} & \SI{5.65}{\EUR} \\
	\textbf{Weighted price} & \textbf{\SI{12.39}{\EUR}}	& \SI{10.21}{\EUR} & \textbf{\SI{5.30}{\EUR}} \\
	\bottomrule
\end{tabu}
\end{table}

%NOW CALCULATE:
% EPSILON_SUB 1.2931034483
% EPSILON_STEAM: 6.44 
% EPSILON: Between 1.2931034483 AND 6.44
% AVG PRICE: Between 5.304493 AND  12.39184 
% M: 3% => 1.03
% CALCULATION:
%
% Case 1: 1.2931034483 * 5.304493 / 1.03 = 6.6594739707
% Case 2: 1.2931034483 * 12.39184 / 1.03 = 15.557214597
% Case 3: 6.44 * 5.304493 / 1.03 = 33.165956233
% Case 4: 6.44 * 12.39184 / 1.03 = 77.479077282

When applying the data of Tbl.~\ref{tab:dataset-stats} and the found $\epsilon$ values to \eqref{eq:computational_efficiency}, it can be inferred that the monthly capacity and licensing cost per peak time user $\mathcal{C}_u$ needs to be below \SI{6.659}{\EUR} (lower bound scenario; $\epsilon_{\text{sub}}$ with $\bar{p}_{\text{min}}$) or EUR \SI{77.479}{\EUR} (upper bound scenario; $\epsilon_{\text{Steam}}$ with $\bar{p}_{\text{min}}$ to obtain a minimum margin $m=1.03$. Scaling theses costs to the peak time population of $11.65\text{m}$ we obtain the maximum server capacity costs $\mathcal{C}_{\text{Cap}}$ between \SI{77.577}{\mega\EUR} and \SI{902.630}{\mega\EUR} per month for a cloud game service of the size of Steam.

Due to the requirement of using special gaming equipment, sharing hardware with other cloud applications seems unrealistic. Thus $\mathcal{C}_u$ can hardly be reduced by cloud service collaboration. However, to lower the investment demands, the platform operator could aim at increasing the overbooking ratio $\epsilon_{sub}$ for the subscription case closer to the $\epsilon_{steam}$ of the classical purchasing model. The provider could, for example, offer off-peak subscriptions that allow the access to the platform only outside of peak hours. 

This maximum further does not consider that the operator may not be able to fully utilize the available capacity or may not hold the optimal game licenses at all times. Thus, in practice, we recommend target $\mathcal{C}_{u}$ values to be lower than the calculate values.

Nevertheless, this cost perspective still points to an interesting observation: due to the low $\epsilon_{sub}$ subscription-based cloud gaming approaches have a higher cost pressure than in the case of a more conventional game sales approach ($\epsilon_{steam}$). When setting the cost perspective in relationship to the product offers, the business challenges of cloud-based game providers become apparent. In particular, the high cost pressures seem to lead to a tightly curated game offering approach --- i.e., few offerings (see Tbl.~\ref{fig:rel-combinedlength-owners}) with good scores (see Fig.~\ref{fig:scores-by-platform}) --- where only economically attractive games can be offered for subscription plans. This is very likely caused by the underlying scale-oriented licensing practices that favor high volumes --- i.e., the initial \gls{CAPEX} is high, but the marginal cost decrease afterwards. The limited game offering for cloud-based services, as a result of cost factors and limited scaling advantages on the hardware side, however, reduces the utility for the customer, which should lower their willingness-to-pay. Obviously, this renders the particularly challenging business environment, in which players go out of business on a regular basis. This further explains why cloud gaming has remained a small niche despite the high interest by industry, research and probably also customers.



%and  $58.3m$ users, we obtain a total server capacity cost $\mathcal{C}_{Cap}$, which should not exceed XXX and XXX Euro resp. 

%the capacity and licensing cost per user $\mathcal{C}_u$ needs to be below XXXXXXXX (for $\epsilon_{max}$)  or XXXX Euro ($\epsilon_{min}$) for obtaining the required minimum margin $m$. 


%Thus, we can characterise that the successful cloud game provider will have a maximum $C_{u}$ in the following bounds:
% for a successful market operation.






%by deploying subscription models that foster the off-peak usage, i.e., off-peak subscriptions that allow the access to the platform only outside of peak hours. This may allow a convergence of $C_u$ for subscription-based charging models close to $\epsilon_{max}$.

%When considering a substantial increase of the $\epsilon$ to $8$---the realistic maximum when considering the high peak time centricity of the gaming use case---, we obtain a substantially lowered $C_{u}$ requirement of $6.44$. 

%\todo[inline]{ADD DATA}




