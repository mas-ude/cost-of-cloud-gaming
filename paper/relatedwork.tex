%!TEX root = paper.tex
%%%%%%%%%%%%%%%%%%%%%%%%%%%%%%%%%%%%%%%%%%%%%%%%%%%%%%%%%%%%%%%%%%%%%%%%%%%%%%%
\section{Related Work}
\label{sec:relatedwork}

As this work aims to investigate Cloud Gaming services, several lines of research are touched. 
First, work that deals with operational benefits of Cloud Gaming and cloud services in general. This also encompasses energy consumption issues.
And second, literature covering \gls{QoE} aspects of gaming in general and Cloud Gaming in particular. These reflect the users' quality expectations but can also outline the requirement for a service's operation.


%%%%%%%%%%%%%%%%%%%%%%%%%%%%%%%%%%%%%%%%%%%%%%%%%%%%%%%%%%%%%%%%%%%%%%%%%%%%%%%
\subsection{Operational and Efficiency Factors}

Unfortunately, many publications on Cloud Gaming only concern themselves with the client's side often even looking at just mobile Cloud Gaming. For example \cite{Soliman2013} overviews some general issues of mobile cloud gaming. Several other publications, e.g. \cite{6924295} and \cite{Huang:2014:MCP:2755535.2755542} investigate the client device's energy saving potential of mobile Cloud Gaming but find rather marginal energy savings.


A 2014 publication \cite{6882299} describes some potential benefits of a centralized cloud gaming platforms, however operates under assumption that can not be uphold under practical circumstances.

Two further papers (\cite{6853364} and \cite{6365107}) suggest an optimization model to place and provision Cloud Gaming \acrshortpl{VM} in order for a service provider to operate at profits. The impact on \gls{QoE} however seems to be significant but is concealed through the lack of absolute data given. The problem of efficiently placing and selecting servers is a pervasive one for Cloud services (e.g. \cite{6740249}), it is however only partially applicable to Cloud Gaming due to the more stringent requirements both in terms of compute resource as well as latency, limiting the selection problem down to relatively narrow geographic regions.


% Also approaches that select (and cost-optimize the selection of) data-centers might not be applicable due to the stringent latency and BW requirements and the need for dedicated hardware (GPUs) and the statefulness of gaming, e.g.
% ``QoS-Aware, Cost-Efficient Selection of Cloud Data Centers'' \cite{6740249}

% ``Cost-efficient Capacitation of Cloud Data Centers for QoS-aware Multimedia Service Provision.'' \cite{hans2014cost} optimization to geographically clustered users


%%%%%%%%%%%%%%%%%%%%%%%%%%%%%%%%%%%%%%%%%%%%%%%%%%%%%%%%%%%%%%%%%%%%%%%%%%%%%%%
\subsection{Gaming QoS/QoE}

In order to compare different gaming services, one needs to first understand how to properly valuate them. For online video games, and Cloud Gaming in particular, a trifecta of objective quality influence factors is most relevant.

First is the concept of \gls{E2E} lag, that is even present in games played entirely offline. It describes the time between phyically triggering an action and seeing it on screen \cite{metzger16lagmodel}. With the presence of an online component and in multiplayer games this lag increases as the remote server needs to verify the actions. Usually be concealed with lag compensation techniques, but these are generally not usable in Cloud Gaming explaining the increased need for regional data centers.

Considerable research efforts have been put into the network delay component of the \gls{E2E} lag, for both online games as well as for Cloud Gaming, though the results vary greatly between each study and between different games. Studies of multiplayer games often focused on \glspl{FPS} such as \textit{Quake 3}~\cite{1266180} or \textit{Unreal Tournament 2003}~\cite{Beigbeder:2004:ELL:1016540.1016556}



``On the Impact of Delay on Real-time Multiplayer Games'' \cite{Pantel:2002:IDR:507670.507674}
%``Latency and Player Actions in Online Games'' \cite{Claypool:2006:LPA:1167838.1167860}
%``Cloud gaming: architecture and performance'' \cite{6574660}

``Measuring the Latency of Cloud Gaming Systems'' \cite{Chen:2011:MLC:2072298.2071991}

``On the Quality of Service of Cloud Gaming Systems'' \cite{6670099}

``The Brewing Storm in Cloud Gaming: A Measurement Study on Cloud to End-user Latency'' \cite{Choy:2012:BSC:2501560.2501563}

``Modeling and Characterizing User Experience in a Cloud Server Based Mobile Gaming Approach'' \cite{5425784}



QoE taxonomy and QoE influences factors for mobile games \cite{beyer2014typedisplaydelayimpact}


sufficiently high image quality of the stream (no noticeable loss of quality compared to local game, even for ``difficult''games with complex scenes or with much noise). Ergo also demand for high throughput for both the data center as well as the user (ideally 15-30Mbit/s per stream)
but also adequately chosen graphical settings; trade-off between resource demands and visual fidelity


sufficiently high frame rate; stuttering; interactivity; but also end-to-end lag influencer, see also \cite{metzger16lagmodel}.




``The Impact of Video Encoding Parameters and Game Type on QoE for Cloud Gaming: a Case Study using the Steam Platform'' \cite{slivarimpact}

``On frame rate and player performance in first person shooter games'' \cite{claypool2007}





subjective tests on the other side

endeavors towards an \acrshort{ITU-T} recommendation for subjective game testing \cite{mollertowards}
``How Sensitive Are Online Gamers to Network Quality?'' \cite{Chen:2006:SOG:1167838.1167859}
``Gaming in the clouds: QoE and the users’ perspective'' \cite{Jarschel20132883}
``Subjective Evaluation of Latency and Packet Loss in a Cloud-Based Game'' \cite{6614351}
``An Evaluation of {QoE} in Cloud Gaming Based on Subjective Tests'' \cite{5976180}



%``A Measurement Study Regarding Quality of Service and Its Impact on Multiplayer Online Games'' \cite{Bredel:2010:MSR:1944796.1944797}
%``Empirical study of subjective quality for Massive Multiplayer Games'' \cite{4604397}



% ``QoE Assessment of Interactivity and Fairness in First Person Shooting with Group Synchronization Control'' \cite{Ida:2010:QAI:1944796.1944806}


%%%%%%%%%%%%
% \subsubsection{Models}
% ``A Comprehensive End-to-End Lag Model for Online and Cloud Video Gaming'' \cite{metzger16lagmodel}


%%%%%%%%%%%%
% \subsubsection{Measurements and Methods}

% ``A Method For Feedback Delay Measurement Using a Low-cost Arduino Microcontroller'' \cite{beyermethod}
% ``Effect of Network Quality on Player Departure Behavior in Online Games'' \cite{4591393}
%``An Empirical Study of Cloud Gaming'' \cite{Manzano:2012:ESC:2501560.2501582}

%%%%%%%%%%%%
% \subsubsection{Studies}


%%%%%%%%%%%%%%%%%%%%%%%%%%%%%%%%%%%%%%%%%%%%%%%%%%%%%%%%%%%%%%%%%%%%%%%%%%%%%%%
% \subsection{QoS/QoE of Cloud Gaming}


%%%%%%%%%%%%
% \subsection{Measurements and Methods}
% ``Using Electroencephalography and Subjective Self-Assessment to Measure the Influence of Quality Variations in Cloud Gaming'' \cite{beyerusing}


%%%%%%%%%%%%
% \subsubsection{Subjective Studies}



%``Are all games equally cloud-gaming-friendly? An electromyographic approach'' \cite{6404025}

% ``Mobile Cloud Gaming: Issues and Challenges'' \cite{Soliman2013}
% short overview of some possible issues

% ``Will Mobile Cloud Gaming Work? Findings on Latency, Energy, and Cost'' \cite{Lampe:2013:MCG:2514943.2515398}
% client-side costs (monetary, energy, latency) of mobile devices; however underestimates energy costs in comparison to local gaming, display is the biggest factor, which is not factored in (this factor is also the same in local as well as cloud gaming); the monetary cost factor also underestimates the contractual situation in countries like Germany
% And despite all this, it still finds mobile cloud gaming rather unfeasible (except in WiFi)

% ``Where Did My Battery Go? Quantifying the Energy Consumption of Cloud Gaming'' \cite{6924295}
% Similar notion as previous paper, more in-depth, focus only on energy costs
% test impl of local running game artificially increases cpu/gpu load
% especially for low load games only marginal energy savings for cloud approach (~12\%)
% even best case (and unrealistically high load) only 38\% energy savings
% considering all the other drawbacks that cloud streaming incurs, this might not be enough

% ``Measuring the Client Performance and Energy Consumption in Mobile Cloud Gaming'' \cite{Huang:2014:MCP:2755535.2755542}
% Similar findings, only 30\% energy savings




%%%%%%%%%%%%
% \subsubsection{QoE Adaptations}

% video rate adaptations
% ``Addressing Response Time and Video Quality in Remote Server Based Internet Mobile Gaming'' \cite{5506572}

% ``Adaptive Mobile Cloud Computing to Enable Rich Mobile Multimedia Applications'' \cite{6413270}


%%%%%%%%%%%%%%%%%%%%%%%%%%%%%%%%%%%%%%%%%%%%%%%%%%%%%%%%%%%%%%%%%%%%%%%%%%%%%%%
% \subsection{Other}


% High frame rates:
% ``The Application of Sampling Theory to Television Frame Rate Requirements''
% \url{http://www.bbc.co.uk/rd/publications/whitepaper282}
% ``High Frame-Rate Television''
% \url{http://www.bbc.co.uk/rd/publications/whitepaper169}
% ``Higher Frame rates for more Immersive Video and Television''
% \url{http://www.bbc.co.uk/rd/publications/whitepaper209}


% Fundamental findings on framerates
% ``Experimentelle Studien über das Sehen von Bewegung'' \cite{wertheimer1912experimentelle}

% VR:
% ``How Do New Visual Immersive Systems Influence Gaming QoE?'' \cite{7148110}


% Hybrid Cloud/Local
% ``Kahawai: High-Quality Mobile Gaming Using GPU Offload'' \cite{Cuervo:2015:KHM:2742647.2742657}


% Speculative Rendering of Future Frames
% ``Outatime: Using Speculation to Enable Low-Latency Continuous Interaction for Mobile Cloud Gaming'' \cite{Lee:2015:OUS:2742647.2742656}

% ``Security issues in online games'' \cite{doi:10.1108/02640470210424455}





