%!TEX root = paper.tex
%%%%%%%%%%%%%%%%%%%%%%%%%%%%%%%%%%%%%%%%%%%%%%%%%%%%%%%%%%%%%%%%%%%%%%%%%%%%%%%
\section{Related Work}
\label{sec:relatedwork}

The existing literature on video game \gls{QoE} is often based on
direct measurements of game and network parameters. However, the
rich variety of video game contents makes it difficult to generalize
the results. This paper's attempt to better understand said variety
and the consumers' view of it is thus an endeavor to support future
studies on classifications of games and gamers.

Considerable research efforts have been put into the network delay component of the \gls{E2E} lag for both online games and cloud gaming. The results, however, remain inconclusive. Studies of multiplayer games often focused on \glspl{FPS} such as \textit{Quake 3}~\cite{1266180} or \textit{Unreal Tournament 2003}~\cite{Beigbeder:2004:ELL:1016540.1016556}. Concerning cloud gaming, Chen et al.~\cite{6670099} for example find very high and variable delay values even when neglecting the network delay. Furthermore, \cite{Choy:2012:BSC:2501560.2501563} gives some insights on the delay requirements of streamed games and the implications for data center distance as well as placement.

Image quality represents a further \gls{QoE} factor. Gaming adds
another dimension to typical image quality assessments, as most
games allow for changes to their graphical fidelity.
%, be it either the resolution or more demanding graphical features, such as ambient occlusion or anti-aliasing. Cloud gaming usually locks these options at one specific setting for a specific quality-to-resource-demand trade-off, resulting in an often lower source quality than what local games can offer.
As an example, the work in \cite{slivarimpact} takes a look at different encoding parameters for cloud gaming.%, but does unfortunately not conduct a full-reference analysis nor investigates the impact of the coding latency.
An overview of some further \gls{QoE} taxonomy and influence factors especially for mobile games is given in \cite{beyer2014typedisplaydelayimpact}. Several efforts also set up subjective tests of cloud gaming services with specific \gls{QoS} parameters in mind. Such studies can be found in, e.g., \cite{Jarschel20132883} and  \cite{6614351}. Efforts have also been made towards an \acrshort{ITU-T} recommendation for subjective game testing as reported in \cite{mollertowards}.

More recently, and much in line with this paper, the \acrshort{ITU-T}
published recommendation G.1032~\cite{itutg1032}, dealing with
influence factors on gaming \gls{QoE}. The factors are grouped by
their source, i.e. the gamer, the hardware/software/network system,
and the context (including the gamer's social embedding, novelty,
and game service factors). Per G.1032's classification, the core of
this paper evaluates the category of context-influencing factors.
