%!TEX root = paper.tex
%%%%%%%%%%%%%%%%%%%%%%%%%%%%%%%%%%%%%%%%%%%%%%%%%%%%%%%%%%%%%%%%%%%%%%%%%%%%%%%%
\section{Conclusion}
\label{sec:conclusion}

While cloud gaming is a topic of strong interest, it comes with a series of problems and limitations, both technical as well as economic in nature. With the conducted user-side analysis it is easy to see the different, curated nature of current cloud gaming services, with their narrow offer of hand-selected games, in the case of \psnow even catered to a specific audience. This limits the attractiveness of any such service, even without looking at more intricate engagement metrics which would require more data than what was available.

But also the operator-side reveals major problems due to the need for highly regional data-centers and special hardware. This eliminates any chance for the efficiency gains that general cloud services are intended for.

However, there might be some niches for specific games or audiences that could be sustained at reduced operational efforts. This might be an angle worth of investigation in the future, especially with better engagement metrics and more detailed models of gaming data center operations.

However, there might also be a simpler and economically more feasible alternative to cloud gaming: game streaming in the local network, which would combine the open market structures of conventional gaming platforms with the ease of use of having a small box in the living that one just can play games on without much thought to the underlying hardware.