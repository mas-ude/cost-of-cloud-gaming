%!TEX root = paper.tex
%%%%%%%%%%%%%%%%%%%%%%%%%%%%%%%%%%%%%%%%%%%%%%%%%%%%%%%%%%%%%%%%%%%%%%%%%%%%%%%%
\section{Conclusion And Future Work}
\label{sec:conclusion}

This paper investigates context influence factors for video gaming
through an online survey among 488 gamers, and an evaluation of
game platform offers from the \steam, \psnow, and \gfnow systems.
The results provide a practical evaluation of influence factors
described in the recent \acrshort{ITU-T} Recommendation
G.1032~\cite{itutg1032}.

The subjective data from the online survey provide an insight into
the gaming demographics of survey participants.
Social context factors influence their buying decisions in the
form of recommendations and news about games, as do technical and
economical aspects.
Respondents buy games mostly from digital storefronts these days,
with \steam as a virtually ubiquitous mention.
Furthermore, Other important service factors, i.e. context factors
that relate to the online platforms offering and operating games,
are investigated.

The objective data are combined from a number of sources:
Game titles are scraped from the platforms (\gfnow, \psnow, and \steam),
game lengths from \hltb, and summarized review scores from
\metacritic.
The datasets used span a timeframe from 2015 to 2017, and will
be made
available from the authors' public repositories.
The utility metrics evaluated include the number (and thus variety)
of games, playtimes, prices, review scores, and combinations of these,
plus some historic aspects.
For example, the \steam ecosystem grew from roughly 6,000 to roughly
14,000 games offered between mid-2015 and late 2017.

Lastly, the survey also asked for subjective assessments of objective
platform properties.






Many other utility metrics are imaginable, e.g.
the motivation to buy games for a platform that the gamer already
possesses,
the number of platform ``exclusive'' game titles,
the game genre and other classifications,
the number of game sales and subscriber numbers,
technical aspects like graphical fidelity, performance, precision
and responsiveness of controls,
measures of the game's content like variety and quality of game mechanics,
or other content-centric factors.

Lastly, any combination of the metrics mentioned above has its
merits for assessing the utility of a gaming platform. For example,
the combined expected playthrough length of the top 10\% of games
across platforms could be evaluated, and judged against the amount
of money required to equip each platform like this.



The dataset presented leaves many possibilities
for future work. For example, many additional investigations of
combined utility metrics and costs per platform are possible.
Furthermore, it was mentioned that utility metrics may be
interpreted differently. If the behavior of prototypical customers
can be estimated from the dataset, then combined and comparable
utility scores could be calculated.

Moreover, direct user studies can complement the data-driven
approach followed in this paper. The utility metrics presented
follow from the personal experience of the authors, but do not
necessarily represent gaming public's opinion. Besides, self-reporting
from game customers can explain conjectured effects like preferential
attachment for popular games.

At any rate, computer gaming and \gls{cg} in particular will
remain interesting topics for research in the years to come,
as new business models and ways to consume games evolve.
