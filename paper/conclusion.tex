%!TEX root = paper.tex
%%%%%%%%%%%%%%%%%%%%%%%%%%%%%%%%%%%%%%%%%%%%%%%%%%%%%%%%%%%%%%%%%%%%%%%%%%%%%%%%
\section{Conclusion And Future Work}
\label{sec:conclusion}

This paper investigates context influence factors for video gaming
through an online survey among 488 gamers, and an evaluation of
game platform offers from the \steam, \psnow, and \gfnow systems.
The results provide a practical evaluation of influence factors
described in the recent \acrshort{ITU-T} Recommendation
G.1032~\cite{itutg1032}.

The subjective data from the online survey provide an insight into
the gaming demographics of survey participants.
Social context factors influence their buying decisions in the
form of recommendations and news about games, as do technical and
economical aspects.
Respondents buy games mostly from digital storefronts these days,
with \steam as a virtually ubiquitous mention.
Furthermore, Other important service factors, i.e. context factors
that relate to the online platforms offering and operating games,
are investigated.

The objective data are combined from a number of sources:
Game titles are scraped from the platforms (\gfnow, \psnow, and \steam),
game lengths from \hltb, and summarized review scores from
\metacritic.
The datasets used span a timeframe from 2015 to 2017, and will
be made
available from the authors' public repositories.
The utility metrics evaluated include the number
of games, playtimes, prices, review scores, and combinations of these,
plus some historic aspects.
For example, the \steam ecosystem grew from roughly 6,000 to roughly
14,000 games offered between mid-2015 and late 2017.

Lastly, the survey also asked for subjective assessments of objective
platform properties. Perhaps unsurprisingly, a large variety of
motivations was reported there. An insight applicable to the size
of gaming offers is that respondents prefer to choose and buy for
themselves rather than having flat-rate access.

Overall, the subjective and objective data about context influence
factors relevant for video gaming presented in this paper are but
one step towards a practical evaluation of these factors.
Many extensions of this study come to mind:
For instance, many other utility metrics, and
particularly combined ones, might be investigated and evaluated
in subjective tests:
the motivation to buy games for a platform that the gamer already
possesses,
the number of platform ``exclusive'' game titles,
the game genre and other classifications,
the number of game sales and subscriber numbers,
technical aspects like graphical fidelity, performance, precision
and responsiveness of controls,
measures of the game's content like variety and quality of game mechanics,
or other content-centric factors.

Second, on the subjective side, online surveys might lead to further
insight into other opinion-based matters of video gaming, and bootstrap
practical in-person studies for further investigation of gaming quality
factors.

\todo[inline]{End on a strong and interesting note.}
