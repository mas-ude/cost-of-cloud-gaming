%!TEX root = paper.tex
%%%%%%%%%%%%%%%%%%%%%%%%%%%%%%%%%%%%%%%%%%%%%%%%%%%%%%%%%%%%%%%%%%%%%%%%%%%%%%%%
\section{Conclusion}
\label{sec:conclusion}

While cloud gaming is a topic of strong interest, it comes with a series of problems and limitations, both of technical and economic nature. With the conducted user-side analysis the curated nature of current cloud gaming services becomes apparent, which entails a narrow offer of hand-selected games and sometimes even a narrow target group (see the case of \psnow). This naturally limits the value of a subscription-based service model for customers, when insufficient convenience and price advantages can be yielded.

%Even without revisiting 
%This limits the attractiveness of any such service, even without looking at more intricate engagement metrics which would require more data than what was available.

But also the operator-side reveals major problems due to the need for highly regional data-centers and special hardware. This eliminates any chance for the efficiency gains that general cloud services are intended for. Moreover, subscription-based models suffer from the expected higher peak utilisation in comparison to à la carte game purchasing models, which creates a high cost pressure due to enormous infrastructure investments and limited scaling advantages. 

However, there might be some niches for specific games or audiences that could be sustained at reduced operational efforts. This might be an angle worth of investigation in the future, especially with better engagement metrics and more detailed models of gaming data center operations. Another economically more feasible alternative to cloud gaming could be game streaming in the local network. This option would still require all the usual gaming hardware and services like \steam but would bring all the convenience and flexibility of cloud gaming.
