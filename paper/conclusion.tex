%!TEX root = paper.tex
%%%%%%%%%%%%%%%%%%%%%%%%%%%%%%%%%%%%%%%%%%%%%%%%%%%%%%%%%%%%%%%%%%%%%%%%%%%%%%%%
\section{Conclusion And Future Work}
\label{sec:conclusion}

\todo[inline]{Check for applicability!}

This paper investigates the prospects of \gls{cg} from a new
perspective: game utility. Complementing the existing literature
on technical and perceptual quality, it
investigates the characteristics of the game catalog
that cloud-based and software distribution gaming
platforms offer, and discusses the customer benefit.
%that a customer might get from these offers.
For this, data from a number of sources are combined and evaluated:
Game titles are scraped from the platforms (\gfnow, \psnow, and \steam),
game lengths from \hltb, and summarized review scores from
\metacritic.
The datasets used span a timeframe from 2015 to 2017, and are
available from the authors' public repositories.
The utility metrics proposed include the number (and thus variety)
of games, playtimes, prices, review scores, and combinations of these,
plus some historic aspects.

The gaming platforms are compared based on these metrics, and a
cost utility model is presented that makes the different pricing
policies (flat rate, subscription with included and surcharged
games, or open à la carte) comparable between platforms.
Currently, \psnow's policy offers the largest number of games
for annual spendings up to approximately $\text{\texteuro} 3,000$. However,
note that the platform has changed its policy
in 2017; it used a subscription/surcharge model like \gfnow
before. Second, \steam's current catalog (while more expensive
assuming the average game price) is 30 times larger than
\psnow's.

The dataset presented leaves many possibilities
for future work. For example, many additional investigations of
combined utility metrics and costs per platform are possible.
Furthermore, it was mentioned that utility metrics may be
interpreted differently. If the behavior of prototypical customers
can be estimated from the dataset, then combined and comparable
utility scores could be calculated.
Also, the \gls{cr} platforms deserve a closer look. Superficially,
they currently serve but a niche market, and combine the negative
aspects of subscriptions, usage-based billing, and per-game costs
for a presumed gain in game \gls{QoE}. It remains to be seen
if there exist less apparent benefits for customers, or if the
\gls{cr} model fails to establish in the market.

Moreover, direct user studies can complement the data-driven
approach followed in this paper. The utility metrics presented
follow from the personal experience of the authors, but do not
necessarily represent gaming public's opinion. Besides, self-reporting
from game customers can explain conjectured effects like preferential
attachment for popular games.

Finally, it would be interesting to investigate the operator-side
issues of \gls{cg} to understand their cost pressure better.
\Gls{cg} requires highly regional data-centers and specialized
hardware to function well. This eliminates many of the efficiency
gains that generic cloud services are intended for.
The efficiency problem is aggravated by high regional peak
(and consequent low average) utilization that require substantial
infrastructure investments at limited scaling advantages.

At any rate, computer gaming and \gls{cg} in particular will
remain interesting topics for research in the years to come,
as new business models and ways to consume games evolve.
